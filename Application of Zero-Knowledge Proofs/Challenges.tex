\section{Challenges Facing Zero-Knowledge Proofs}

We saw how ZKPs can be implemented in different areas and more precisely in Blockchain. Despite their numerous advantages, Zero-Knowledge Proofs face several challenges that need to be addressed to facilitate their widespread adoption. These challenges include computational efficiency, trusted setup, scalability, and regulatory considerations.

\subsection{Computational Efficiency}

One of the main challenges with ZKPs is their computational overhead. Generating and verifying zero-knowledge proofs can be computationally expensive, especially for large-scale systems. Although advances like zk-SNARKs and zk-STARKs have made ZKPs more efficient, generating proofs can still be resource-intensive, particularly for complex computations or large datasets.
\\
For example, in privacy-centric blockchain systems, the computational resources required to generate zk-SNARK proofs can be significant, which can slow down transaction processing. Similarly, verifiable computations in cloud services may require additional resources, increasing costs and computation time.
\\
To address this, researchers are exploring more efficient ZKP schemes, such as Bulletproofs, which reduce proof size and verification time without compromising security.

\subsection{Trusted Setup}

Some ZKP protocols, particularly zk-SNARKs, require a trusted setup phase to generate initial parameters that are used to construct proofs. If the parameters are not generated securely, the entire system can be compromised, potentially allowing malicious actors to forge proofs.
\\
The trusted setup problem poses a significant challenge, as users need to trust that the setup phase was performed honestly. One solution is to use multi-party computation (MPC) ceremonies, where multiple parties contribute randomness to the setup process. However, this adds complexity to the system and may still leave users uneasy about the security of the parameters.
\\
Newer ZKP schemes, such as zk-STARKs, do not require a trusted setup, making them more transparent and secure in environments where trust is difficult to establish.

\subsection{Scalability}

While ZKPs offer strong privacy guarantees, their scalability remains a concern. As more applications adopt ZKP-based solutions, the need for more scalable proof systems becomes essential. In blockchain systems, for example, the growing number of transactions and users can lead to longer proof generation and verification times, limiting throughput.
\\
Scaling ZKP protocols to support large-scale applications without compromising security is a significant challenge. Techniques like recursive proofs, where a proof can attest to the correctness of previous proofs, are being explored to improve scalability. Additionally, layer-2 solutions such as zk-Rollups help mitigate scalability concerns by batching transactions, but further advancements are necessary to support widespread adoption.

\subsection{Post-Quantum Security}

With the advent of quantum computing, many cryptographic systems, including some ZKP schemes, may become vulnerable to quantum attacks. ZKPs that rely on elliptic curve cryptography, such as zk-SNARKs, could potentially be broken by quantum computers. This poses a long-term challenge for ZKP systems that need to be future-proof.
\\
Fortunately, zk-STARKs, which are based on hash functions rather than elliptic curve cryptography, are believed to be resistant to quantum attacks. However, ensuring the post-quantum security of all ZKP systems remains an active area of research.

\subsection{Regulatory and Legal Concerns}

The use of ZKPs in financial systems, privacy-focused cryptocurrencies, and digital identity verification raises regulatory and legal concerns. Governments and regulatory bodies may have concerns about ZKPs enabling illicit activities, such as money laundering or tax evasion, due to their ability to hide transaction details.
\\
Balancing the privacy benefits of ZKPs with the need for regulatory oversight is a significant challenge. Policymakers must carefully consider how to regulate ZKP-based systems without undermining their privacy-preserving features. Additionally, legal frameworks must evolve to accommodate new cryptographic technologies and ensure that they are used responsibly.